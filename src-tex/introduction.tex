
The aim of this book is to provide instruction and reflection on Buddhist meditation as taught by Ajahn Sumedho, using material extracted from talks he gave in the early 1980s. These talks were almost all given to monastics who were familiar with the language and terms of Theravada Buddhism – but Ajahn Sumedho's approach is not technically intricate, and so we felt that many more people could benefit from these instructions than the small gatherings in the monasteries. You are therefore invited to make use this book for your own spiritual practice.

I have added a brief section at the outset for beginners; the book then continues with Ajahn Sumedho's introduction to meditation. Part two is a collection of practical instructions. The third part of the book offers an example of how the understanding that meditation develops can be applied to our everyday lives.

The first edition of this book, titled `\textit{Path to the Deathless},' was printed in 1985 to coincide with the opening of Amaravati (`Deathless Realm') Buddhist Centre. The centre has subsequently been redefined as a monastery, and the book given its current title to highlight `mindfulness,' a prominent feature of Buddhist meditation. Mindfulness, the simple faculty of bearing a chosen theme, sensation or aspiration in mind, is now widely respected in therapeutic as well as spiritual circles. It is the simplicity of mindfulness that makes Buddhist meditation accessible, while the effects of paying attention to one's inner life are far-reaching and profound.

Mindfulness is taught in many different ways, but in this book, and through a teaching career of more than 35 years, Ajahn Sumedho has introduced it in a way that encourages personal enquiry and reflection rather than a highly technical system. He fleshes out this approach with details of his own practice and the good humour born of kindness and non-attachment.

Just as the book's first edition opened Amaravati, this fourth edition comes shortly after Ajahn Sumedho has retired from being the abbot there. He would however hasten to add, using the words of the Buddha: `the Doors of the Deathless are open, let those who can hear bring forth their faith.' The teachings and the Path remain.

Friends at Aruna Ratanagiri have been instrumental in bringing this edition of the book into a revised form by creating a digital version and tidying up the design. Sash Lewis has given the text an editorial scanning and made grammatical and typographical improvements that have encouraged a further review of the text. I have updated some of the examples that Ajahn Sumedho used (which were current in the 1980s) to those that are more timeless. The main thrust of the teachings is unchanged.

May the efforts of all who have been involved in the production of this book be for their long-lasting welfare and happiness.

\bigskip
{\par\raggedleft
Ajahn Sucitto\\
Cittaviveka Buddhist Monastery\\
May 2012
\par}

