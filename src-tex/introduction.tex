
The aim of this book is to provide a clear instruction in and reflection on Buddhist meditation as taught by Ajahn Sumedho, a bhikkhu (monk) of the Therav\=adin tradition. The following chapters are edited from longer talks Ajahn Sumedho has given to meditators as a practical approach to the wisdom of Buddhism. This wisdom is otherwise known as Dhamma, or `the way things are.'

You are invited to use this book as a step-by-step manual. The first chapter tries to make the practice of meditation clear in a general way and the subsequent sections can be taken one at a time and followed by a period of meditation. The third chapter is a reflection on the understanding that meditation develops. The book concludes with the means of taking the Refuges and Precepts which place the practice of meditation within the larger framework of mind-cultivation. These can be requested formally from ordained Buddhists (Sa\.ngha) or personally determined. They form the foundation of the means whereby spiritual values are brought into the world.

The first edition of this book (2,000 copies) was printed in 1985 -- for the opening of the Amaravati Buddhist Centre -- and stocks were quickly exhausted. People appreciated the book, and some asked to help sponsor a re-print; so we gave the manuscript a more thorough proof-reading than had been possible before, and added some design to improve the `feel' of the book -- otherwise the text is the same. As this book is entirely produced by voluntary contributions and acts of service to the Dhamma, readers are asked to respect this offering and make it freely available.

May all beings realise Truth.

\bigskip
{\par\raggedleft
Ajahn Sucitto\\
Amaravati Buddhist Centre\\
May 1986
\par}

