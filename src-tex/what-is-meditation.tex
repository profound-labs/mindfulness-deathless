

 
The word meditation is a much used word these days, covering a wide range of practices. In Buddhism it designates two kinds of meditation -- one is called `\textit{samatha}', the other `\textit{vipassana}.' Samatha meditation is one of concentrating the mind on an object, rather than letting it wander off to other things. One chooses an object such as the sensation of breathing, and puts full attention on the sensations of the inhalation and exhalation. Eventually through this practice you begin to experience a calm mind -- and you become tranquil because you are cutting off all other impingements that come through the senses.

The objects that you use for tranquillity are tranquillising (needless to say!). If you want to have an excited mind, then go to something that is exciting, don't go to a Buddhist monastery, go to a disco! \ldots{} Excitement is easy to concentrate on, isn't it? It's so strong a vibration that it just pulls you right into it. You go to the cinema and if it is really an exciting film, you become enthralled by it. You don't have to exert any effort to watch something that is very exciting or romantic or adventurous. But if you are not used to it, watching a tranquillising object can be terribly boring. What is more boring than watching your breath if you are used to more exciting things? So for this kind of ability, you have to arouse effort from your mind, because the breath is not interesting, not romantic, not adventurous or scintillating - it is just as it is. So you have to arouse effort because you're not getting stimulated from outside.

In this meditation, you are not trying to create any image, but just to concentrate on the ordinary feeling of your body as it is right now: to sustain and hold your attention on your breathing. When you do that, the breath becomes more and more refined, and you calm down \ldots{} I know people who have prescribed samatha meditation for high blood pressure because it calms the heart.

So this is tranquillity practice. You can choose different objects to concentrate on, training yourself to sustain your attention till you absorb or become one with the object. You actually feel a sense of oneness with the object you have been concentrating on, and this is what we call absorption.

The other practice is 'vipassana', or 'insight meditation'. With insight meditation you are opening the mind up to everything. You are not choosing any particular object to concentrate on or absorb into, but watching in order to understand the way things are. Now what we can see about the way things are, is that all sensory experience is impermanent. Everything you see, hear, smell, taste, touch; all mental conditions - your feelings, memories and thoughts - are changing conditions of the mind, which arise and pass away. In vipassana, we take this characteristic of impermanence (or change) as a way of looking at all sensory experience that we can observe while sitting here.

This is not just a philosophical attitude or a belief in a particular Buddhist theory: impermanence is to be insightfully known by opening the mind to watch, and being aware of the way things are. It's not a matter of analysing things by assuming that things should be a certain way and, when they aren't, then trying to figure out why things are not the way we think they should be. With insight practice, we are not trying to analyse ourselves or even trying to change anything to fit our desires. In this practice we just patiently observe that whatever arises passes away, whether it is mental or physical.

So this includes the sense organs themselves, the object of the senses, and the consciousness that arises with their contact. There are also mental conditions of liking or disliking what we see, smell, taste, feel or touch; the names we give them; and the ideas, words and concepts we create around sensory experience. Much of our life is based on wrong assumptions made through not understanding and not really investigating the way anything is. So life for one who isn't awake and aware tends to become depressing or bewildering, especially when disappointments or tragedies occur. Then one becomes overwhelmed because one has not observed the way things are.

In Buddhist terms we use the word Dhamma, or Dharma, which means 'the way it is', 'the natural laws'. When we observe and 'practise the Dhamma', we open our mind to the way things are. In this way we are no longer blindly reacting to the sensory experience, but understanding it, and through that comprehension beginning to let go of it. We begin to free ourselves from just being overwhelmed or blinded and deluded by the appearance of things. Now to be aware and awake is not a matter of becoming that way, but of being that way. So we observe the way it is right now, rather than doing something now to become aware in the future. We observe the body as it is, sitting here. It all belongs to nature, doesn't it? The human body belongs to the earth, it needs to be sustained by the things that come out of the earth. You cannot live on just air or try to import food from Mars and Venus. You have to eat the things that live and grow on this Earth. When the body dies, it goes back to the earth, it rots and decays and becomes one with the earth again. It follows the laws of nature, of creation and destruction, of being born and then dying. Anything that is born doesn't stay permanently in one state, it grows up, gets old and then dies. All things in nature, even the universe itself, have their spans of existence, birth and death, beginning and ending. All that we perceive and can conceive of is change; it is impermanent. So it can never permanently satisfy you.

In Dhamma practice, we also observe this unsatisfactoriness of sensory experience. Now just note in your own life that when you expect to be satisfied from sensory objects or experiences you can only be temporarily satisfied, gratified maybe, momentarily happy - and then it changes. This is because there is no point in sensory consciousness that has a permanent quality or essence. So the sense experience is always a changing one, and out of ignorance and not understanding, we tend to expect a lot from it. We tend to demand, hope and create all kinds of things, only to feel terribly disappointed, despairing, sorrowful and frightened. Those very expectations and hopes take us to despair, anguish, sorrow and grief, lamentation, old age, sickness and death.

Now this is a way of examining sensory consciousness. The mind can think in abstractions, it can create all kinds of ideas and images, it can make things very refined or very coarse. There is a whole gamut of possibilities from very refined states of blissful happiness and ecstasies to very coarse painful miseries: from Heaven to Hell, using more picturesque terminology. But there is no permanent Hell and no permanent Heaven, in fact no permanent state that can be perceived or conceived of. In our meditation, once we begin to realise the limitations, the unsatisfactoriness, the changing nature of all sensory experience, we also begin to realise it is not me or mine, it is `\textit{anatt\=a}', not-self.

So, realising this, we begin to free ourselves from identification with the sensory conditions. Now this is done not through aversion to them, but through understanding them as they are. It is a truth to be realised, not a belief. 'Anatta' is not a Buddhist belief but an actual realisation. Now if you don't spend any time in your life trying to investigate and understand it, you will probably live your whole life on the assumption that you are your body. Even though you might at some moment think, 'Oh, I am not the body', you read some kind of inspired poetry or some new philosophical angle. You might think it is a good idea that one isn't the body, but you haven't really realised that. Even though some people, intellectuals and so forth, will say, 'We are not the body, the body is not self', that is easy to say, but to really know that is something else. Through this practice of meditation, through the investigation and understanding of the way things are, we begin to free ourselves from attachment. When we no longer expect or demand, then of course we don't feel the resulting despair and sorrow and grief when we don't get what we want. So this is the goal -- `\textit{Nibb\=ana},' or realisation of non-grasping of any phenomena that have a beginning and an ending. When we let go of this insidious and habitual attachment to what is born and dies, we begin to realise the Deathless.

Some people just live their lives reacting to life because they have been conditioned to do so, like Pavlovian dogs. If you are not awakened to the way things are, then you really are merely a conditioned intelligent creature rather than a conditioned stupid dog. You may look down on Pavlov's dogs that salivate when the bell rings, but notice how we do very similar things. This is because with sensory experience it is all conditioning, it is not a person, it is no 'soul' or 'personal essence'. These bodies, feelings, memories and thoughts are perceptions conditioned into the mind through pain, through having been born as a human being, being born into the families we have, and the class, race, nationality; dependent on whether we have a male or female body, attractive or unattractive, and so forth. All these are just the conditions that are not ours, not me, not mine. These conditions, they follow the laws of nature, the natural laws. We cannot say, 'I don't want my body to get old' - well, we can say that, but no matter how insistent we are, the body still gets old. We cannot expect the body to never feel pain or get ill or always have perfect vision and hearing. We hope, don't we? 'I hope I will always be healthy, I will never become an invalid and I will always have good eyesight, never become blind; have good ears so I will never be one of those old people that others have to yell at; and that I will never get senile and always have control of my faculties 'til I die at ninety-five, fully alert, bright, cheerful, and die just in my sleep without any pain.' That is how we would all like it. Some of us might hold up for a long time and die in an idyllic way, tomorrow all our eyeballs might fall out. It is unlikely, but it could happen! However, the burden of life diminishes considerably when we reflect on the limitations of our life. Then we know what we can achieve, what we can learn from life. So much human misery comes out of expecting a lot and never quite being able to get everything one has hoped for.

So in our meditation and insightful understanding of the way things are, we see that beauty, refinement, pleasure are impermanent conditions -- as well as pain, misery and ugliness. If you really understand that, then you can enjoy and endure whatever happens to you. Actually, much of the lesson in life is learning to endure what we don't like in ourselves and in the world around us; being able to be patient and kindly, and not make a scene over the imperfections in the sensory experience. We can adapt and endure and accept the changing characteristics of the sensory birth and death cycle by letting go and no longer attaching to it. When we free ourselves from identity with it, we experience our true nature, which is bright, clear, knowing; but is not a personal thing anymore, it is not 'me' or 'mine' - there is no attainment or attachment to it. We can only attach to that which is not ourself!

The Buddha's teachings are merely helpful means, ways of looking at sensory experience that help us to understand it. They are not commandments, they are not religious dogmas that we have to accept or believe in. They are merely guides to point to the way things are. So we are not using the Buddha's teachings to grasp them as an end in themselves, but only to remind ourselves to be awake, alert and aware that all that arises passes away.

This is a continuous, constant observation and reflection on the sensory world, because the sensory world has a powerfully strong influence. Having a body like this with the society we live in, the pressures on all of us are fantastic. Everything moves so quickly - television and the technology of the age, the cars -- everything tends to move at a very fast pace. It is all very attractive, exciting and interesting, and it all pulls your senses out. Just notice when you go to London how all the advertisements pull your attention out to whiskey bottles and cigarettes! Your attention is pulled into things you can buy, always going towards rebirth into sensory experience. The materialistic society tries to arouse greed so you will spend your money, and yet never be contented with what you have. There is always something better, something newer, something more delicious than what was the most delicious yesterday \ldots{} it goes on and on and on, pulling you out into objects of the senses like that.

But when we come into the shrine room, we are not here to look at each other or to be attracted or pulled into any of the objects in the room, but to use them for reminding ourselves. We are reminded to either concentrate our minds on a peaceful object, or open the mind, investigate and reflect on the way things are. We have to experience this, each one for ourselves. No-one's enlightenment is going to enlighten any of the rest of us. So this is a movement inwards: not looking outwards for somebody who is enlightened to make you enlightened. We give this opportunity for encouragement and guidance so that those of you who are interested in doing this can do so. Here you can, most of the time, be sure that nobody is going to snatch your purse! These days you can't count on anything, but there is less risk of it here than if you were sitting in Piccadilly Circus; Buddhist monasteries are refuges for this kind of opening of the mind. This is our opportunity as human beings.

As a human being we have a mind that can reflect and observe. You can observe whether you are happy or miserable. You can observe the anger or jealousy or confusion in you mind. When you are sitting and feel really confused and upset, there is that in you which knows it. You might hate it and just blindly react to it, but if you are more patient you can observe that this is a temporary changing condition of confusion or anger or greed. But an animal cannot do that; when it is angry it is completely that, lost in it. Tell an angry cat to watch its anger! I have never been able to get anywhere with our cat, she cannot reflect on greed. But I can, and I am sure that the rest of you can. I see delicious food in front of me, and the movement in the mind is the same as our cat Doris's. But we can observe the animal attraction to things that smell good and look good.

This is using wisdom by watching that impulse, and understanding it. That which observes greed is not greed: greed cannot observe itself, but that which is not greed can observe it. This observing is what we call 'Buddha' or 'Buddha wisdom' -- awareness of the way things are.
